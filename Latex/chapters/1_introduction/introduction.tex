Nonostante gli enormi passi avanti che sono stati compiuti nel campo delle tecnologie usate nelle reti di calcolatori, sono ancora molteplici i casi in cui Internet non trasmette le informazioni alle velocità sperate. \bigskip

Eppure molto s’investe nella ricerca e sviluppo, di un’infrastruttura efficiente per la trasmissione dei segnali, che garantisca elevati livelli di delivery rate end-to-end (in ogni normale condizione). \bigskip

Come dunque, si può immaginare, la causa principale non è nè l’uso di tecnologie inadeguate, nè la mancanza di conoscenza nel loro uso. Il reale freno alle potenziali velocità raggiungibili, nasce da una scelta effettuata negli anni '80, quando fu definito il primo schema per il controllo di congestione del protocollo TCP. \bigskip

Tale scelta riguardava l’identificare il fenomeno di congestione (in senso stretto) con il fenomeno di perdita dei pacchetti. \bigskip

Tuttavia, nonostante i due concetti siano molto differenti, al tempo tale scelta aveva senso, solo in virtù delle limitazioni tecnologiche. \bigskip

Oggi anche la migliore versione loss-based di TCP: CUBIC, fa fatica a garantire un delivery rate elevato. Basti pensare che per poter sostenere 10Gbps con un RTT di 100ms, è necessario un tasso di perdita dello 0.000003\%, e con lo stesso RTT ed un tasso di perdita dell’ 1\% sono sostenibili solo 3Mbps \cite[p.~12]{ietf:draft-ietf-tcpm-cubic-05}. \bigskip

Lo scopo di quest'elaborato è dunque quello di presentare un'alternativa per tali approcci, che non sia più basata sull'evento di packet loss, ma esplicitamente sull'evento di congestione. L'alternativa che è stata approfondita è il nuovo controllo di congestione BBR. \bigskip

Prima di iniziare, è lecito far presente che lo sviluppo di tale soluzione è ancora in corso, per cui alcuni aspetti di questa prima versione potrebbero subire delle modifiche nel corso del tempo.