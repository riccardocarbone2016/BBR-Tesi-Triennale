Il comportamento di BBR è dettato da un modello del path di rete, su cui il flusso di pacchetti viaggia. Tale modello traccia l'andamento nel corso del tempo del \textit{BtlBw} e dell'\textit{RTprop}. \bigskip

Guidato dal proprio modello il BBR setta i propri parametri di controllo :

\begin{enumerate}

\item \texttt{pacing\_rate}: è il parametro principale dell'algoritmo, e regola il sending rate del mittente nel corso del tempo;

\item \texttt{send\_quantum}: definisce la massima unità di aggregazione per i pacchetti da trasmettere. Utile per diminuire l'overhead della CPU del destinatario.

\item \texttt{cwnd}: non ricopre un ruolo principale, ma continua ad essere il limite superiore per la quantità di dati inflight possibile del sender. 

\end{enumerate}

L'apposita configurazione di tali parametri permetterà al BBR di operare nel punto operativo migliore (target operating point). Ovvero quello in cui sono soddisfatte le condizioni di :

\begin{itemize}

\item \textit{Rate balance} : il sending rate del mittente ed il bottleneck rate sono perfettamente bilanciati;

\item \textit{Full pipe} : la quantità di dati inflight è uguale al \textit{BDP} corrente. La pipe opera al 100\%, nulla più, nulla meno;

\end{itemize}

\begin{nota}{Nota: focus sull'installazione del BBR}

Tutto ciò che successivamente diremo dell'algoritmo BBR, riguarda solo il lato mittente TCP. Non è necessario alcun cambiamento nel destinatario TCP per utilizzare quest'algoritmo di controllo di congestione. \bigskip

E' importante però mettere l'accento sul fatto che ciò valga per il TCP che già fa uso di un sistema di ARQ (automatic repeat request). \bigskip

Fondamentalmente il BBR potrebbe essere utilizzato anche da un altro protocollo di livello trasporto, l'importante è che disponga di un sistema di ARQ, in quanto gli ack sono alla base del corretto funzionamento dell'algoritmo.

\end{nota}

\section{Il network path model}

In questa sezione approfondiamo come vengono misurati i parametri caratteristici del network path model.

\subsection{Bottleneck bandwidth} 

Nella regione di bandwidth limited illustrata nel grafico \ref{Delivery_rate_round_trip_time}, il delivery rate è praticamente pari al bottleneck rate, essendo questo il suo limite superiore. Quindi la stima del \textit{BtlBw} può essere raggiunta attraverso il campionamento del delivery rate. \bigskip

Il tempo con la quale sono generati i campioni utili alla stima è temporizzato dagli ACK ricevuti dal mittente.
Approfondiamo adesso il processo di generazione dei campioni (per una trattazione specifica consultare \cite{ietf:draft-cheng-iccrg-delivery-rate-estimation-00}). \bigskip

\subsubsection{L'ack rate}

Un ottimo indice possibile per il delivery rate è l'ack rate, ovvero il tasso di dati riscontrati istantaneamente dal destinatario. \bigskip

Per il suo calcolo è necessario mantenere delle informazioni di stato, per ogni pacchetto, e per la connessione:

\begin{enumerate}

\item \texttt{C.delivered}: quantità di dati riscontrati, fino all’istante \texttt{C.delivered\_time};

\item \texttt{C.delivered\_time}: istante in cui è pervenuto l’ack più recente;

\item \texttt{P.delivered}: valore di \texttt{C.delivered} all’atto della trasmissione di P;

\item \texttt{P.delivered\_time}: valore di \texttt{C.delivered\_time} all’atto della trasmissione di P;

\end{enumerate}

A partire da queste è possibile definire l'espressione dell'ack rate:

\begin{center}

\textit{ack rate \texttt{= (C.delivered - P.delivered) / (C.delivered\_time - P.delivered\_time)}}

\end{center}

E' fondamentale comprendere che l'ack rate è vicino al delivery rate, se la quantità di dati riscontrata nel $ \Delta t$ individuato corrisponde a ciò che effettivamente il mittente trasferisce a destinazione.\bigskip

Fenomeni come la compressione, o soppressione degli ack potrebbero alterare la loro normale temporizzazione, portando a sovrastimare o sottostimare (rispettivamente) il delivery rate \cite{ietf:ack-suppression}. \bigskip

La compressione è il principale nemico, in quanto una sovrastima per ragioni chiare avanti, è più pericolosa di una sottostima. \bigskip

\begin{approfondimento}{Approfondimento: significato matematico dell'ack rate}

Mostriamo una possibile sequenza di trasmissione, da parte di un generico sender:

\begin{figure}[H]

\center
\caption{Ack rate determinato in corrispondenza del riscontro per Pkt 3}
\includegraphics[scale=0.3]{chapters/3_bbr/img/ack_rate}

\end{figure}

All'atto della ricezione del riscontro per il pacchetto 3, viene determinato l'ack rate come:

\begin{center}

\textit{ack rate = (Pkt3.size + Pkt2.size) / (3s - 1s)}

\end{center}

Dunque l'ack rate rappresenta la pendenza del segmento evidenziato. Ovvero la pendenza con la quale cresce il tasso di dati riscontrati. \bigskip

In aggiunta si fa riflettere che il risultato (considerando pacchetti di dimensione fissa) è 1 pacchetto/secondo. Ed intuitivamente è ciò che ci saremo aspettati come delivery rate. (Ecco perchè l'ack rate è un buon indice in condizioni normali) \bigskip 

\end{approfondimento}

\subsubsection{Il send rate}

Per valutare la veridicità dell'ack rate viene introdotto il send rate. \bigskip

Ancora una volta per la sua determinazione abbiamo bisogno di mantenere delle informazioni di stato (ulteriori alle precedenti), per ogni pacchetto:

\begin{enumerate}

\item \texttt{P.first\_sent\_time}: istante in cui è iniziata la trasmissione dell’ultimo pacchetto riscontrato dall’ack ricevuto al tempo \texttt{P.delivered\_time} (a monte dell’invio di P) 
\item \texttt{P.sent\_time}: istante in cui parte la trasmissione del pacchetto

\end{enumerate}

A partire da queste definiamo l'espressione del send rate:

\begin{center}
 
\textit{send rate = \texttt{ (C.delivered - P.delivered) / (P.sent\_time - P.first\_sent\_time)}}

\end{center}

La differenza con l'ack rate sta nell'intervallo di tempo considerato. Il send time rappresenta il tempo che è trascorso tra la trasmissione dell'ultimo pacchetto riscontrato, ed il pacchetto attualmente riscontrato. \bigskip

Questo permette che il $\Delta t$ considerato includa il tempo di trasmissione dei dati riscontrati, in quanto il send time è in generale >= dell'effettivo tempo di trasmissione. Così è possibile evitare il problema delle sovrastime dell'ack rate. \bigskip

\begin{esempio}{Esempio: un caso di compressione degli ack}

Mostriamo una possibile sequenza di trasmissione, in cui si verifica una compressione degli ack ricevuti:

\begin{figure}[H]

\center
\caption{Fenomeno di compressione degli ack}
\includegraphics[scale=0.9]{chapters/3_bbr/img/ack_rate_overstimate.pdf}

\end{figure}

Come si nota dalla figura gli ack per i pacchetti 2-200 vengono ritardati e compressi, ed arrivano in burst tra la trasmissione del pacchetto 201 ed il relativo ack. \bigskip

Ciò porta l'ack rate ad essere 200 packets/s, ma si comprende bene che tra 3s e 4s il sender non ha trasferito 200 pacchetti a destinazione. Quindi in tale situazione l'ack rate non è un indice veritiero del delivery rate. \bigskip

Il reale tempo che i 200 pacchetti impiegano per arrivare al destinatario è di 2s. \bigskip

In questo caso il send rate è di 60 packets/s, perchè considera il tempo trascorso dalla trasmissione del pacchetto 1 a quella del pacchetto 201. Il secondo in più è dovuto all'inclusione del secondo tra 2 e 3 secondi, in cui il mittente è inattivo. \bigskip

Confrontando quindi il send rate e l'ack rate si comprende che il destinatario non può aver ricevuto 200 pacchetti tra 3s e 4s.

\end{esempio}

\begin{approfondimento}{Approfondimento: significato matematico del send rate}

Consideriamo la stessa sequenza di trasmissione precedente da parte di un generico sender:

\begin{figure}[H]

\center
\caption{Send rate determinato in corrispondenza del riscontro per Pkt 4}
\includegraphics[scale=0.3]{chapters/3_bbr/img/send_rate}

\end{figure}

All'atto della ricezione del riscontro per il pacchetto 4 viene determinato il send rate come : \bigskip

\begin{center}

\textit{send rate = (Pkt3.size + Pkt4.size) / (2s - 0s)}

\end{center}

Dunque il send rate rappresenta la pendenza del segmento evidenziato in verde. Ovvero la pendenza con la quale cresce il tasso di dati trasmessi. \bigskip 

\end{approfondimento}

\subsubsection{Stima del \textit{BtlBw}}

Ad ogni ack sarà generato un campione relativo al delivery rate come segue:

\begin{lstlisting}[caption=Delivery rate evaluation]
	time_elapsed = max(send_time,ack_time)
	delivery_rate = data_acked / time_elapsed
\end{lstlisting}

Non ci resta altro che scoprire come da tali campioni venga ottenuta una stima del bottleneck rate. \bigskip

Per poter far questo BBR usa un \textit{BtlBw}MaxFilter, un windowed filter di ampiezza $ W_{B} $ :

\[
	\widehat{\textit{BtlBw}_{T}} \: = \: max(deliveryRate_{t}) \; \forall t\: \in \: [T-W_{B},T]  
\]

L'ampiezza $ W_{B} $ è misurata in multipli di \textit{RTprop}, e deve essere:

\begin{itemize}

\item Abbastanza estesa da contenere un ProbeBW cycle gain (sezione \ref{ProbeBW}). Così che essa contenga dei sample raccolti nella regione di bandwidth limited, che è effettivamente l'unica regione in cui la stima potrà essere corretta;

\item Abbastanza estesa da combattere l'eventuale presenza di rumore, introdotta da eventuali code lungo il percorsa;

\item Abbastanza corta da poter permettere una pronta a reazione ad una diminuzione del bottleneck rate;

\end{itemize} 

Un ottimo valore è costituito da 10 \textit{RTprop}s \cite[p.~61]{Cardwell:2017:BCC:3042068.3009824}.

\subsection{Round Trip propagation delay}

Per la stima dell'\textit{RTprop} siamo aiutati dal fatto che il protocollo TCP già effettua un campionamento dell'RTT, per la gestione dei timeout di ritrasmissione. \bigskip

Anche per determinare \textit{RTprop} BBR usa un windowed filter, \textit{RTprop}MinFilter di ampiezza $ W_{R} $ :

\[
	\widehat{\textit{RTprop}_{T}} \: = \: min(RTT_{t}) \; \forall t\: \in \: [T-W_{R},T]  
\]

L'ampiezza $ W_{R} $ è misurata in secondi, e deve essere :

\begin{itemize}

\item Estesa almeno quanto un ProbeRTTInterval (sezione \ref{ProbeRTT}). Così che essa contenga dei sample raccolti nella regione di application limited, che è effettivamente l'unica regione in cui la stima potrà essere corretta;

\item Abbastanza estesa da evitare una consistente diminuzione del delivery rate raramente (sezione \ref{ProbeRTT});

\item Abbastanza estesa da poter sfruttare gli istanti in cui l'applicazione limita il delivery rate del mittente, per poter aggiornare la stima dell'\textit{RTprop} senza diminuire fortemente il delivery rate;

\item Abbastanza corta da poter permettere una pronta reazione ad un variazione dei ritardi;

\end{itemize}

\section{I parametri di controllo}

Compreso in primo luogo come il BBR mantenga aggiornato il modello del network path. Vediamo ora di approfondire i parametri di controllo, usati da quest'ultimo per portarsi nel target operating point.

\subsection{Pacing rate}

Il pacing rate del BBR costituisce un parametro importantissimo (e primario), con cui vengono incontrate nel tempo le condizioni di \textit{Rate balance}, e di \textit{Full pipe}. \bigskip

Effettuare il pacing dei pacchetti significa equi-distanziare le loro trasmissioni. Quindi a monte dell'invio di un dato pacchetto il successivo sarà trasmesso al tempo:

\begin{center}

\textit{nextSendTime = now + packet.size / pacing rate}

\end{center}

Ciò porta ad un adattamento del tasso di trasmissione del mittente al pacing rate, nonostante la capacità del proprio link d'ingresso possa essere superiore. La spaziatura definisce dunque le pause adatte per sincronizzare il sending rate come voluto. \bigskip

Il set di tale parametro (così come gli altri) avviene ad ogni ack, e può essere riassunto con il seguente pseudocodice:

\begin{lstlisting}[caption=BBRSetPacingRate]
BBRSetPacingRate():
	BBRSetPacingRateWithGain(BBR.pacing_gain)

BBRSetPacingRateWithGain(pacing_gain):
	rate = pacing_gain * BBR.BtlBw
	if (BBR.filled_pipe || rate > BBR.pacing_rate)
		BBR.pacing_rate = rate

\end{lstlisting}

Da cui come ci si aspettava il pacing rate è settato proporzionalmente al \textit{BtlBw}, attraverso il pacing gain.
La condizione dell'if non ne permette il decremento in condizioni di pipe underfilled. \bigskip

Un valore di pacing gain pari 1 può permettere di operare nel target operating point, se non sono state formate code in precedenza. Un valore di pacing gain > 1 è utile per ottenere sample utili alla stima del \textit{BtlBw}, in quanto viene saturato il delivery rate. Ed infine un valore di pacing gain < 1 può essere utilizzato per svuotare le code generate, ed uscire da una situazione di pipe overfill.

\subsection{Send quantum}

Rappresenta la massima unità di aggregazione per i pacchetti inviati dal mittente. Permette quindi d'inviare gruppi di pacchetti in un unico burst, diminuendo l'overhead della CPU del destinatario. \bigskip

Settato ad ogni ack, il suo valore dipende fortemente dal pacing rate. Per valori alti di quest'ultimo è possibile usare alti valori di send quantum, mentre per valori bassi è meglio diminuirlo, in quanto creerebbe un aumento dei ritardi e delle code lungo il percorso.

\subsection{Congestion window}

La cwnd impone un limite alla quantità di dati inflight del mittente. Nonostante il parametro principale del BBR per il controllo del sending rate sia il pacing rate, la cwnd è molto utilizzata nelle situazioni in cui si vuole subito portare il limite inflight in un particolare punto. Verrebbe speso un maggior tempo se si usufruisse del pacing rate. \bigskip

Per la sua gestione vengono definiti un upper ed un lower bound:

\begin{itemize}

\item \textit{target cwnd}: limite superiore della cwnd, ottenuto in funzione del $ \widehat{\textit{BDP}} $ (\textit{BDP} stimato) come:

\begin{center}
\textit{target cwnd = cwnd gain * $ \widehat{\textit{BDP}} $}
\end{center}

\item \textit{BBRMinPipeCwnd}: limite inferiore della cwnd, tipicamente tale da consetire l'invio di 2/4 pacchetti al mittente.

\end{itemize}

Tali limiti sono da considerare strettamente rigidi in condizioni normali (non occorrono nè eventi di loss, nè di timeout). \bigskip

Discutiamo adesso della modulazione del parametro in tali condizioni. \bigskip

Se la pipe fosse piena, è molto probabile che si sia raggiunto il target, dunque si provvede ad incrementare la cwnd solo se essa non supererà il target. In caso contrario resterà fissata su quest'ultimo. \bigskip 

Se invece fossimo al di sotto del target, la cwnd subirebbe un semplice incremento pari alla quantità di dati consegnati a destinazione. \bigskip

In ogni caso viene sempre garantito il rispetto del lower bound. \bigskip

Tuttavia gli eventi di packet loss e di timeout portano ad una modulazione diversa della cwnd. \bigskip

Ad un evento di timeout, il BBR porta la cwnd cautamente ad una SMSS, in quanto viene assunto perso tutto ciò che è stato inviato. Da qui poi si ripartirà come descritto in precedenza. \bigskip

In occorrenza invece di un evento di packet loss, la cwnd viene subito (al primo round) adeguata al delivery rate attuale come segue:

\begin{lstlisting}[caption=BBR enter in loss recovery]

BBR.prior_cwnd = BBRSaveCwnd()
cwnd = packets_in_flight + max(packets_delivered, 1)
BBR.packet_conservation = true

\end{lstlisting}

Adeguandosi poi, nelle fasi successive, sempre al delivery rate attuale:

\begin{lstlisting}[caption=BBRModulateCwndForRecovery]

BBRModulateCwndForRecovery():
	if (packets_lost > 0)
   		cwnd = max(cwnd - packets_lost, 1)
	if (BBR.packet_conservation)
    	cwnd = max(cwnd, packets_in_flight + packets_delivered)

\end{lstlisting}

Dopo un round trip time in Loss Recovery, si ripristina la cwnd al suo precedente valore (salvato con BBRSaveCwnd):

\begin{lstlisting}[caption=BBR exit from loss recovery]
BBR.packet_conservation = false
BBRRestoreCwnd()

BBRRestoreCwnd():
	cwnd = max(cwnd, BBR.prior_cwnd)

\end{lstlisting}

A questo punto riportiamo il codice completo di gestione della congestion window:

\begin{lstlisting}[caption=BBRSetCwnd]
BBRSetCwnd():
	BBRUpdateTargetCwnd()
	BBRModulateCwndForRecovery()
	if (not BBR.packet_conservation) {
   		if (BBR.filled_pipe)
        	cwnd = min(cwnd + packets_delivered, BBR.target_cwnd)
		else if (cwnd < BBR.target_cwnd || BBR.delivered < InitialCwnd)
			cwnd = cwnd + packets_delivered
		cwnd = max(cwnd, BBRMinPipeCwnd)
    }
	BBRModulateCwndForProbeRTT()

\end{lstlisting}

\section{Visione complessiva dell'algoritmo}

In questa sezione fonderemo tutti gli aspetti visti in precedenza. Comprendendo affondo come lavora un flusso BBR nel corso della sua vita. \bigskip

Il miglior punto di partenza è visualizzare gli stati in cui il flusso può trovarsi con uno state chart diagram:

\begin{figure}[H]

\center
\caption{BBR state diagram}
\includegraphics[scale=0.6]{chapters/3_bbr/img/state_diagram}

\end{figure}

Ricordiamo che gli obiettivi della state machine sono:

\begin{itemize}

\item Raggiungere il punto di massimo per il delivery rate, ed il punto di minimo per l'RTT;

\item Mantenere continuamente aggiornato il network path model, stimando sequenzialmente nel corso del tempo, il bottleneck rate ed il round trip propagation delay;

\item Garantire una equa suddivisione della capacità di rete tra flussi BBR concorrenti;

\end{itemize}

Passiamo ora all'analisi degli stati e delle transizioni.

\subsection{Startup}

Durante la fase di Startup, così com'era per il classico Slow Start, è d'interesse scovare subito il limite massimo del delivery rate. \bigskip

Ora considerando gli standard attuali è ragionevole avere a che fare con 9 ordini di grandezza (1 Gbps), 10 ordini o 11 ordini di grandezza (10/100 Gbps). \bigskip

Pertanto è necessario che progressivamente ad ogni round, il sending rate perlomeno raddoppi. Ciò viene ottenuto configurando il pacing gain, ed il cwnd gain ad un valore pari al BBRHighGain (2/ln(2)). Ovvero la quantità di guadagno minima che garantisce un raddoppio del tasso del mittente:

\begin{lstlisting}[caption=BBREnterStartup]
BBREnterStartup():
	BBR.state = Startup
	BBR.pacing_gain = BBRHighGain
	BBR.cwnd_gain = BBRHighGain

\end{lstlisting}

La fase di Startup termina nel momento in cui si riempie completamente la pipe. Ciò che tipicamente accade, è che non si riesce ad arrivare precisamente nel punto operativo, ma un pò aldilà, generando così delle code di una quantità pari a (cwnd gain - 1) * \textit{BDP}. \bigskip

Per individuare poi in che momento si è saturato il delivery rate, viene usato il network path model.
Ciò che si fa è attendere che per 3 round di fila il bottleneck rate stimato non cresca più del 25\%. Quindi viene ricercato fondamentalmente un plateau per quest'ultimo. \bigskip

Motiviamo poi perchè siano necessari 3 round. Il primo round serve al destinatario per effettuare l'auto-tuning della finestra di ricezione, sulla base dell'attuale capacità di rete. Il secondo round è usato dal mittente per riempire la finestra di ricezione. Il terzo è quello in cui arriveranno i riscontri per i pacchetti inviati, e cioè i sample che effettivamente sono indicativi del delivery rate.

\subsection{Drain}

Dopo una fase di Startup entriamo in Drain. Lo scopo di tale fase è svuotare le code formate in precedenza. Ciò viene ottenuto in un round invertendo il valore del pacing gain (così da lavorare con un sending rate < bottleneck rate):

\begin{lstlisting}[caption=BBREnterDrain]
BBREnterDrain():
	BBR.state = Drain
	BBR.pacing_gain = 1/BBRHighGain
	BBR.cwnd_gain = BBRHighGain

\end{lstlisting}

La fase di drain termina nel momento in cui la quantità di dati inflight arriva al valore attuale stimato del \textit{BDP}. Quindi le code saranno tutte svuotate. \bigskip

A questo punto si passa in ProbeBW.

\subsection{ProbeBW} \label{ProbeBW}

Il BBR arriva nello stato ProbeBW senza la presenza di code sul path di comunicazione. \bigskip

E' in questo stato in cui il BBR spende la maggior parte del suo tempo. \bigskip

In ProbeBW il BBR stima il bottleneck rate, operando in condizioni di massimo delivery rate e ritardo di propagazione minimo. \bigskip

Ciò viene ottenuto attraverso una tecnica chiamata gain cycling. Un gain cycle è tipicamente costituito da 8 fasi, in ognuna di queste fasi il pacing gain viene impostato ad un valore utile a raggiungere le condizioni sopra espresse. \bigskip

I valori di pacing gain usati per ogni fase sono i seguenti: [5/4,3/4,1,1,1,1,1,1]. \bigskip

Partiamo dalla fase 0. In questa fase il pacing gain porta il mittente ad operare con un pacing rate > \textit{BtlBw}. Questo serve ad ottenere campioni utili alla stima di quest'ultimo, in quanto è solo nella regione di bandwidth limited (figura \ref{Delivery_rate_round_trip_time}) ove il delivery rate $ \simeq $ bottleneck rate. \bigskip

La fase 0 dura almeno un \textit{RTprop}, e termina nel momento in cui la quantità di dati inflight è pari a 5/4*\textit{BDP}, o si verificano degli eventi di loss (non è possibile raggiungere la condizione inflight >= 5/4*\textit{BDP}). \bigskip

Dopo la fase 0, l'obiettivo della fase 1 è drenare le code che sono state formate, impostando un valore di pacing rate di 3/4. Dunque in un tempo pari a \textit{RTprop}, o prima (se le code si svuotano rapidamente) il flusso BBR torna nuovamente nel target point. \bigskip

Dopodichè dalla fase 2 in poi, quest'ultimo opera nelle condizioni ideali con un pacing gain pari a 1. \bigskip 

Proponiamo di seguito uno scenario reale di funzionamento del cycle gaining, in cui il bottleneck rate è di 10Mbps:

\begin{figure}[H]

\center
\caption{BBR probe bw gain cycling}
\includegraphics[scale=1.2]{chapters/3_bbr/img/probe_bw_gain_cycling}
\caption*{Figura tratta da \cite[p.~62]{Cardwell:2017:BCC:3042068.3009824}}

\end{figure}   
   
Aggiungiamo solo che in questo stato la cwnd non ha un ruolo centrale, ed il suo target viene impostato a 2*\textit{BDP}. Questo per non disturbare la modulazione del pacing rate, e per avere comunque un limite di sicurezza per l'inflight amout.

\subsection{ProbeRTT} \label{ProbeRTT}

Il BBR usa tale stato per poter ottenere sample dell'RTT utili a stimare correttamente l'\textit{RTprop}. \bigskip

Ogni ProbeRTTInterval (di una durata tipica di 10 s) il BBR transita in ProbeRTT se la stima non è stata aggiornata. In tale stato ci si serve della cwnd per dare un colpo netto all quantità di dati inflight, impostandola al suo limite minimo (BBRMinCwndPipe). In questo modo il flusso opera nella regione di application limited, ove si otterrano sample dell'RTT $ \simeq $ \textit{RTprop}. \bigskip

La cwnd è mantenuta a BBRMinCwndPipe per un tempo pari al ProbeRTTDuration, che è circa 200 ms. Questa scelta permette di spendere solo il 2\% del tempo in ProbeRTT, dove il delivery rate è limitato fortemente, a discapito del 98\% speso giustamente in ProbeBW. \bigskip

Allo scadere di un ProbeRTTDuration viene ripristinata la cwnd, e si transita o in ProbeBW, o in Startup, a seconda dello stato attuale della rete.

\subsection{Evoluzione di un flusso BBR}

In conclusione possiamo dire che l'evoluzione di un flusso BBR, passa per un:

\begin{enumerate}

\item \textit{Warm up}: si passa in stati non stabili come Startup e Drain per scovare velocemente il limite della banda di trasmissione;

\item \textit{Stati stabili}: quali ProbeBW e ProbeRTT, in cui il BBR aggiorna il proprio network path, ed imposta i propri parametri per raggiungere il target operating point;

\end{enumerate}

\begin{figure}[H]

\center
\caption{BBR state evolution}
\includegraphics[scale=0.8]{chapters/3_bbr/img/bbr_evolution}

\end{figure}   
