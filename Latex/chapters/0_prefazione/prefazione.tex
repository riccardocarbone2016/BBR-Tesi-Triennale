
La scelta dell'argomento trattato all'interno dell'elaborato (il controllo di congestione BBR) è ricaduta su una delle tematiche che sono state affrontate durante il progetto \#{WikiTIM}, svoltosi nel periodo di Giugno-Luglio 2017 nell'ambito di una collaborazione tra l'Università di Napoli Federico II, Wikimedia Italia e TIM. \bigskip

Lo scopo di tale progetto era, ed è, quello di avvicinare il mondo dell'università alla più grande enciclopedia libera del Web: Wikipedia. E' infatti fondamentale per una maggiore valorizzazione del patrimonio informativo che essa contiene, far sì che ogni sua voce non sia solo chiara, ma corredata di riferimenti bibliografici a fonti attendibili ed accessibili a tutti. \bigskip

Il progetto \#{WikiTIM} ha avuto come risultato finale la valorizzazione delle voci relative al: controllo di congestione, ed alla gestione di rete. \bigskip

In particolare, il gruppo di lavoro dell'Università Federico II di Napoli ha contribuito alla revisione delle voci:

\begin{itemize}

\item Controllo di congestione: \\ 
\url{https://it.wikipedia.org/wiki/Controllo_della_congestione}

\item Controllo di congestione in TCP: \\
\url{https://it.wikipedia.org/wiki/Controllo_della_congestione_in_TCP}

\item Gestione di rete: \\
\url{https://it.wikipedia.org/wiki/Gestione_di_rete}

\item Simple Network Management Protocol: \\
\url{https://it.wikipedia.org/wiki/Simple_Network_Management_Protocol}

\end{itemize}  

