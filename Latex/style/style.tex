%%%%%%%%%%%%%%%%%%%%%%%%%%%%%%%%%%%%%%%%%%%%%%%%%%%%%%%%%%%%%
%%%%%%                       Style                     %%%%%%
%%%%%%%%%%%%%%%%%%%%%%%%%%%%%%%%%%%%%%%%%%%%%%%%%%%%%%%%%%%%%

%------------------------------------------------------------
% Margini

\geometry{top=30mm,bottom=30mm}

%------------------------------------------------------------
% Stile titoli dei capitoli

%\titlespacing{\chapter}
%{0pt}
%{0pt}%{-\headsep-11pt}
%{3\baselineskip}
%
%\titleformat{\chapter}[frame]
%{\normalfont}
%{\filright
%\footnotesize
%\enspace \thechapter\enspace}
%{8pt}
%{\Large\bfseries\filcenter}

\titlespacing{name=\chapter} {0pt}{50pt}{20pt}

\titleformat
{name=\chapter}                  % command
[display]                        % shape
{\normalfont\huge\bfseries}      % format
{} % label
{0pt}                           % sep
{\LARGE}[\vspace{1ex}\titlerule]  % before-code    %% adjust 2ex here as you want.

%------------------------------------------------------------
% Stile titoli delle sezioni

%\titleformat{\section}[block]
%{\normalfont%\sffamily
%}
%{\thesection}{\large}{\titlerule\\[.8ex]\bfseries}

\titleformat
{name=\section}                  % command
[block]                        % shape
{\normalfont\large\bfseries}      % format
{\thesection\enspace} % label
{0pt}                           % sep
{\large}[\vspace{1ex}]  % before-code    %% adjust 2ex here as you want.

%------------------------------------------------------------
% Stile titoli delle sotto-sezioni

\titleformat
{name=\subsection}                  % command
[block]                        % shape
{\normalfont\normalsize\bfseries}      % format
{\thesubsection\enspace} % label
{0pt}                           % sep
{\normalsize}[\vspace{1ex}]  % before-code    %% adjust 2ex here as you want.

%------------------------------------------------------------
% Stile dei collegamenti

\hypersetup{colorlinks=true,
			filecolor= gray,
			linkcolor= gray,
			urlcolor= gray,
			citecolor=gray,			
			linktocpage}

%------------------------------------------------------------
% Enunciati

\theoremstyle{plain}
\newtheorem{thm}{Theorem}[section]
\newtheorem{cor}[thm]{Corollario}
\newtheorem{lem}[thm]{Lemma}
\newtheorem{prop}[thm]{Proposition}
\theoremstyle{definition}
\newtheorem{defn}{Definition}[chapter]
\theoremstyle{remark}
\newtheorem{oss}{Osservation}

\newenvironment{sistema}%
{\left\lbrace\begin{array}{@{}l@{}}}%
{\end{array}\right.}

%------------------------------------------------------------
% Stile di pagina delle sezioni

\pagestyle{fancy}
\renewcommand{\chaptermark}[1]{\markboth{\emph{\small{Capitolo \thechapter}}}{}}
\renewcommand{\sectionmark}[1]{\markright{\emph{\small{\thesection\ \boldmath#1\unboldmath}}}}
\fancyfoot[C]{\small\thepage}
%\renewcommand{\footrulewidth}{0pt} Ampiezza della linea del piè
\newcommand{\firstpagenumber}{
\clearpage{\pagestyle{empty}\cleardoublepage}
}
\renewcommand{\headrulewidth}{1pt} %Ampiezza della linea della testatina
%\fancyhead[L]{\leavevmode\smash{\protect\includegraphics[scale=0.7]{Immagini/logo}}}
%\fancyhead[R]{\tiny Titolo}

%------------------------------------------------------------
% Ridefinizione dello stile plain

\fancypagestyle{plain}{%
\renewcommand{\headrulewidth}{0pt} %Ampiezza della linea della testatina
\fancyhf{}
\fancyfoot[C]{\small\thepage}}

%------------------------------------------------------------
% Livello di numerazione delle sezioni

\setcounter{secnumdepth}{5}

%------------------------------------------------------------
% Stile delle caption

\captionsetup[table]{labelfont={bf},textfont={small},
justification={raggedright}}
\captionsetup[figure]{labelfont={bf},textfont={small},
justification={raggedright}}

%------------------------------------------------------------
% Stile dei codici (Python)

\definecolor{mygreen}{rgb}{0,0.6,0}
\definecolor{mygray}{rgb}{0.5,0.5,0.5}
\definecolor{mymauve}{rgb}{0.58,0,0.82}

\lstset{ %
  backgroundcolor=\color{white},   % choose the background color; you must add \usepackage{color} or \usepackage{xcolor}; should come as last argument
  basicstyle=\small,        % the size of the fonts that are used for the code
  breakatwhitespace=false,         % sets if automatic breaks should only happen at whitespace
  breaklines=true,                 % sets automatic line breaking
  captionpos=t,                    % sets the caption-position to bottom
  commentstyle=\color{mygreen},    % comment style
  deletekeywords={...},            % if you want to delete keywords from the given language
  escapeinside={\%*}{*)},          % if you want to add LaTeX within your code
  extendedchars=true,              % lets you use non-ASCII characters; for 8-bits encodings only, does not work with UTF-8
  frame=single,	                   % adds a frame around the code
  keepspaces=true,                 % keeps spaces in text, useful for keeping indentation of code (possibly needs columns=flexible)
  keywordstyle=\color{blue},       % keyword style
  language=Octave,                 % the language of the code
  morekeywords={*,...},           % if you want to add more keywords to the set
  numbers=left,                    % where to put the line-numbers; possible values are (none, left, right)
  numbersep=5pt,                   % how far the line-numbers are from the code
  numberstyle=\tiny\color{mygray}, % the style that is used for the line-numbers
  rulecolor=\color{black},         % if not set, the frame-color may be changed on line-breaks within not-black text (e.g. comments (green here))
  showspaces=false,                % show spaces everywhere adding particular underscores; it overrides 'showstringspaces'
  showstringspaces=false,          % underline spaces within strings only
  showtabs=false,                  % show tabs within strings adding particular underscores
  stepnumber=2,                    % the step between two line-numbers. If it's 1, each line will be numbered
  stringstyle=\color{mymauve},     % string literal style
  tabsize=2,	                   % sets default tabsize to 2 spaces
  title=\lstname                   % show the filename of files included with \lstinputlisting; also try caption instead of title
}

% Nomi convenzionali di Codici ed Elenco dei codici
\addto\captionsitalian{%
	\renewcommand{\lstlistingname}{Codice}}
\addto\captionsitalian{%
	\renewcommand{\lstlistlistingname}{Elenco dei codici}}
	
%------------------------------------------------------------
% Stile dei colori

\definecolor{shadecolor}{gray}{0.95}

%------------------------------------------------------------
% Stile delle boxes

\newtcolorbox
{approfondimento}[1]{colback=gray!20,
colframe=gray!80,fonttitle=\bfseries,
title=#1}

\newtcolorbox
{esempio}[1]{colback=yellow!20,
colframe=yellow!50!red!65,fonttitle=\bfseries,
title=#1}

\newtcolorbox
{suggerimento}[1]{colback=blue!20,
colframe=blue!75,fonttitle=\bfseries,
title=#1}

\newtcolorbox
{nota}[1]{colback=green!15,
colframe=green!50!black!80,fonttitle=\bfseries,
title=#1}

%------------------------------------------------------------

